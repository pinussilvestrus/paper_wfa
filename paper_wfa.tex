\documentclass[conference]{IEEEtran}
\IEEEoverridecommandlockouts
% The preceding line is only needed to identify funding in the first footnote. If that is unneeded, please comment it out.
\usepackage[utf8]{inputenc}
\usepackage[T1]{fontenc}
\usepackage[ngerman]{babel}
\usepackage{cite}
\usepackage{amsmath,amssymb,amsfonts}
\usepackage{algorithmic}
\usepackage{graphicx}
\usepackage{textcomp}
\usepackage[colorinlistoftodos,prependcaption,textsize=tiny]{todonotes}
\def\BibTeX{{\rm B\kern-.05em{\sc i\kern-.025em b}\kern-.08em
    T\kern-.1667em\lower.7ex\hbox{E}\kern-.125emX}}
\begin{document}

\selectlanguage{ngerman}
\title{Einschätzung der Fahrtauglichkeit eines Fahrers mit Wearables oder Smartphone}

\author{\IEEEauthorblockN{Niklas Kiefer}
\IEEEauthorblockA{\textit{Hochschule Harz} \\
Wernigerode, Deutschland \\
u33505@hs-harz.de}}

\maketitle

\begin{abstract}
\todo[inline]{Todo!}
\end{abstract}

\begin{IEEEkeywords}
driving fitness, smartphones, smartwatches
\end{IEEEkeywords}

\section{Einleitung}
\label{introduction}
Das Entstehen von Verkehrsunfällen, die auf fahruntüchtige Verkehrsteilnehmer zurückzuführen sind, ist ein großes Problem für die Sicherheit im Straßenverkehr. Es gibt zahlreiche Richtlinien und psychologische Tests, um die Fahrtauglichkeit eines Fahrers zu überprüfen \cite{drivervehiclelicencingagency,testverfahrenpsychometrischefahreignung,begutachtungsrichtlinien}\todo{mehrere Zitate in einer Klammer?}. Diese bestehen jedoch aus langwierigen Untersuchungen, so dass eine finale Einschätzung erst nach einer bestimmten Zeit getroffen werden kann. \todo{wichtig (vor Fahrtantritt)! noch später darauf eingehen}  Eine schnelle Entscheidung vor Fahrtantritt ist damit ohne Weiteres nicht möglich.
\section{Hintergrund und verwandte Arbeiten}
\label{relatedWork}
\todo[inline]{Abschnitt vielleicht doch einfach nur 'Related Works' nennen?}
\todo[inline]{Klare Abgrenzung zu den Related Work}
Dieser Abschnitt beschäftigt sich mit der Aufbereitung artverwandter Arbeiten zum Thema Einschätzung der Fahrtauglichkeit. Zuerst wird ein Überblick über gängige Richtlinien in der Verkehrspsychologie gegeben. Dann wird ein kurzer Überblick über bereits entwickelter Software in diesem Bereich gegeben. 
\section{Feststellung geeigneter Parameter}
\todo[inline]{besserer Titel!}
\section{Analyse der Datenvalidität}
\label{dataValidity}
Nachdem im Abschnitt \ref{parameters} eine Reihe von geeigneten Parametern ermittelt wurde, soll die Qualität der aus den darauf beruhenden Messmöglichkeiten untersucht werden.

\subsection{Kategorie App-Tests Smartphone (PS)}

Die einzelnen Parameter werden im folgenden nacheinander mithilfe verschiedener Quellen auf Qualität überprüft. Auf die Aufgabenstellung einzelner Tests wird in dieser Arbeit nicht eingegangen. 

Laut der Studie von Kramm ist der 1992 entwickelte Cognitrone-Test (\textit{PS01}) zuverlässig in der Bestimmung der Aufmerksamkeit - und Konzentrationsleistung \cite{studieaufmerksamkeitstests}. Die Studie von Neelima schätzt die Vertrauenswürdigkeit der Testergebnisse eher durchschnittlich ein \cite{indiaassessment}. Mit einer Dauer von ca. 6-10 Minuten sei er zudem relativ kurz. Die Schwierigkeit des Tests sei mit zunehmenden Alter jedoch als sehr hoch einzuschätzen. In Hinsicht zur Situation des Fahrers, dass etwaige Tests schnell vor dem Fahrtantritt bewältigt werden müssen, ist eine kurze Testdauer und eine ausgewogene Schwierigkeit essentiell. Dieses Problem wird im Abschnitt \ref{openChallenges} noch thematisiert.

Parameter \textit{PS02} kann u.a. mit dem vom Albrecht et al. entwickelten System untersucht werden \cite{mobilesmarttracking}. Hier sei es möglich, mithilfe der Kamera eines Smartphones einen Nystagmus des Auges festzustellen. Die Ergebnisse seien sehr vielversprechend. Durch soll ein Augenzittern ist ein verminderndes Sehvermögen des Fahrers durchaus beurteilbar. Anderseits beschreiben Albrecht et al. ebenfalls, dass das aufgezeigte System als Teil einer umfänglicheren Testbatterie, und nicht als einziges Indiz genutzt werden sollte \cite{mobilesmarttracking}. Als weitere Testmaßnahme zum Sehvermögen kann indirekt der Linienverfolgungstest gewählt werden. Eine schlechte Sehleistung könnte bei diesem dazu führen, dass Linienverläufe falsch erkannt werden. Somit könnte man durch ein schlechtes Testergebnis auf eine schlechte Sehleistung folgen. Die Ermittlung des Sehvermögens bleibt jedoch kritisch in diesem Kontext. Zwar gibt es indirekte Methoden zur Bestimmung, jedoch führt selbst die Wiener Testreihe keinen eindeutigen Test zur direkten Messung.

Das räumliche Vorstellungsvermögen (\textit{PS03}) könnte mit dem Adaptiven Dreidimensionalen Würfeltest getestet werden. Laut der Studie von Bennett et al. hat der Test auch eine gute Zuverlässigkeit \cite{cognitivetestsfitnesstodrive}. Jedoch hat der Test eine Länge zwischen 29 und 52 Minuten\footnote{Schuhfried.at. (2018). SCHUHFRIED - A3DW. [online] Verfügbar unter: https://www.schuhfried.at/test/A3DW [Zugriff 1 Feb. 2018]} und wäre in dieser Form nicht einsetzbar. Der Intelligenz-Basis-Funktionen-Test prüft die räumliche Vorstellungskraft ebenfalls indirekt über einzelne Items. Somit kommt es auf die einzelnen Fragen innerhalb dieses Test an, um gezielt den Parameter PS03 zu überprüfen. Jedoch fassen Bennett et al. in ihrer Studie die Aussagekraft eines solchen Intelligenztest als nicht so hoch ein \cite{cognitivetestsfitnesstodrive}. PS03 ist somit mit genannten Test-Methoden eher nicht valide und wieder indirekt zu überprüfen.

Parameter \textit{PS04} kann mithilfe des Visuellen Gedächtnistests überprüft werden. In ähnlichen Ausführungen, wie z.B. dem Salford Objective Recognition Test, wird ein solcher Test in der Arbeit von Bennett et al. als vertrauenswürdig eingeschätzt \cite{cognitivetestsfitnesstodrive}.

Die Belastbarkeit (\textit{PS05})  kann unter anderem mit dem Wiener Determinationstest überprüft werden. Die Studie von Neelima schätzt die Validtät der Testergebnisse des DT auf gut bis durchschnittlich ein \cite{indiaassessment}. Schuhfried et al. haben eine umfassende Evaluierung dieses Tests vorgenommen und dessen Validität sehr detailliert bestätigt \cite{wiendt}. Der Test habe in seiner Kurform nur eine Dauer von 5 Minuten und keine hohe Komplexität, sei aber trotzdem sehr zuverlässig in der Messung der Belastbarkeit.

Der Parameter \textit{PS06} schließt neben der Anpassungsfähigkeit die verkehrstechnische Entscheidungsfähigkeit mit ein. Für beide Eigenschaften ist der ATAVT entwickelt wurden, um Situationen im Straßenverkehr zu präsentieren und die Beobachtungsfähigkeit des Fahrers einzuschätzen. Die Länge des Tests ist auf mindestens 8 Minuten\textsuperscript{\ref{foot:atavt}}  einzuordnen und somit relativ lang, um vor Fahrtantritt eingesetzt zu werden. Zudem bewertet Neelima in ihrer Studie die Validität des ATAVT eher durchschnittlich \cite{indiaassessment}.

Der Trail-Making-Test  ist laut der Studie von Bennett et al. eines der international renommiertesten Tests für das Messen der Reaktionsfähigkeit (\textit{PS07}) \cite{cognitivetestsfitnesstodrive}. Ebenfalls haben Baker et al. festgestellt, dass dieser Test sehr häufig in anderen Studien benutzt  und zudem als sehr zuverlässig eingeschätzt wird \cite{reviewofassessmenttests}. Zudem dauert der Test nur wenige Minuten und wäre somit auf dem Smartphone sehr einfach implementierbar und vor Fahrtantritt schnell ausführbar. Der Wiener Reaktionstest ist eine weitere Möglichkeit, um die Reaktionsfähigkeit zu testen. Er ist mit 5-10 Minuten Testdauer\footnote{Schuhfried.at. (2018). SCHUHFRIED - RT. [online] Verfügbar unter: https://www.schuhfried.at/test/RT [Zugriff 1 Feb. 2018]} relativ kurz. Das einfache, schnelle Drücken von Farbpunkten auf dem Smartphone-Display wäre keine große Implementierungsaufgabe.

\todo[inline]{restliche Parameter}

\todo[inline]{Aufzählen, welche Parameter gut zu messen sind}

\subsection{Kategorie Smart-Tracking Wearables (PW)}
\section{Konzept zur Datenerfassung und - auswertung}
\label{concept}
Ein weiteres wichtiges Thema dieser Arbeit soll es sein, ein erstes Konzept zur automatischen Datenerfassung und - auswertung der nötigen Parameterdaten zu entwickeln. Dieses kann dann als Vorlage für einen Prototypen zur automatischen Einschätzung der Fahrtauglichkeit eines Fahrers vor Fahrtantritt dienen. Der grundlegende Entwurf ist in der Abbildung \ref{fig:conceptfmc} als FMC-Modell dargestellt.

\begin{figure}
	\centering
	\includegraphics[width=\linewidth]{images/ConceptDriverAssessmentData}
	\caption[Caption for concept]{FMC-Modell der grundlegenden Anwendung}
	\label{fig:conceptfmc}
\end{figure}
\section{Diskussion der Ergebnisse}
\label{evaluation}
Im Anschluss an das entwickelte Konzept werden die Ergebnisse nun evaluiert und kritisch diskutiert. 
\input{chapters/discussion}
\section{Zusammenfassung}
\label{conclusion}
Zusammenfassend lässt sich ohne weiteres sagen, dass die Einschätzung der Fahrtauglichkeit eines Fahrers mithilfe von Smartphones und Smartwatches möglich ist. Die Ergebnisse dieser Arbeit haben gezeigt, das gerade vor Fahrtantritt eine Reihe von Parametern gemessen werden können ... todo

\todo[inline]{Eigene Literatur einpflegen!}

\begin{thebibliography}{00}

\bibitem{mobilesmarttracking} U. Albrecht, K. Khosravianarab, K. Folta-Schoofs, J. Teske, J. Kanngießer, and U. von Jan, 'Mobile Smarttracking – Finding objective parameters for determining fitness to drive', Biomedical Engineering / Biomedizinische Technik, Berlin, 2013
\bibitem{smartphoneresearchplatform} M. C. Nefzger, 'Design and Development of a Low-Cost Smartphone-Based Research Platform for Real-World Driving Studies', Ludwig-Maximilians-Universität, München, 2017
\bibitem{wearabletracking} F. El-Amrawy, and M. I. Nounou, 'Are Currently Available Wearable Devices for Activity Tracking and Heart Rate Monitoring Accurate, Precise, and Medically Beneficial?', Alexandria University, Alexandria, 2015
\bibitem{fitnesstrackingbajpaj}  A. Bajpai, V. Jilla, V. N. Tiwari, S. M. Venkatesan, and R. Narayanan, 'Quantifiable fitness tracking using wearable devices,' 2015 37th Annual International Conference of the IEEE Engineering in Medicine and Biology Society (EMBC), Milan, 2015
\bibitem{monitoringstressheartrate} S. Mayya, V. Jilla, V. N. Tiwari, M. M. Nayak, and R. Narayanan, 'Continuous monitoring of stress on smartphone using heart rate variability', 2015 IEEE 15th International Conference on Bioinformatics and Bioengineering (BIBE), Belgrade, 2015
\bibitem{selfmonitoringheartrate} E. E. Dooley 'Measuring the Validity of Self-monitoring Heart Rate and Activity Tracking Wearables', University of Texas, Austin, 2016
\bibitem{biofeedbackwearables} E. Jo, K. Lewis, D. Directo , M. J.  Kim, and B.A. Dolezal. 'Validation of Biofeedback Wearables for Photoplethysmographic Heart Rate Tracking', Journal of Sports Science \& Medicine, 2016
\bibitem{activityrecognition} J. R. Kwapisz, G. M. Weiss, and S. A. Moore, 'Activity Recognition using Cell Phone Accelerometers', Fordham University, New York, 2010
\bibitem{drivervehiclelicencingagency} Driver \& Vehicle Licensing Agency, 'Assessing fitness to drive - a guide for medical professionals', DVLA, Swansea, 2017
\bibitem{activityrecognition} S. Dernbach, B. Das, N. C. Krishnan, B. L. Thomas, and D. J. Cook, 'Simple and Complex Activity Recognition through Smart Phones', 2012 Eighth International Conference on Intelligent Environments, Guanajuato, 2012
\bibitem{securityprivacyfitnesstracking} W. Zhou and S. Piramuthu, 'Security/privacy of wearable fitness tracking IoT devices', 2014 9th Iberian Conference on Information Systems and Technologies (CISTI), Barcelona, 2014
\bibitem{citekey}
\end{thebibliography}

\end{document}
