\section{Analyse der Datenvalidität}
\label{dataValidity}
Nachdem im Abschnitt \ref{parameters} eine Reihe von geeigneten Parametern ermittelt wurde, soll die Qualität der aus den darauf beruhenden Messmöglichkeiten untersucht werden.

\subsection{Kategorie App-Tests Smartphone (PS)}

Die einzelnen Parameter werden im folgenden nacheinander mithilfe verschiedener Quellen auf Qualität überprüft. Auf die Aufgabenstellung einzelner Tests wird in dieser Arbeit nicht eingegangen. 

Laut der Studie von Kramm ist der 1992 entwickelte Cognitrone-Test (\textit{PS01}) sehr zuverlässig in der Bestimmung der Aufmerksamkeit - und Konzentrationsleistung \cite{studieaufmerksamkeitstests}. Mit einer Dauer von ca. 6-10 Minuten sei er zudem relativ kurz. Die Schwierigkeit des Tests sei mit zunehmenden Alter jedoch als sehr hoch einzuschätzen. In Hinsicht zur Situation des Fahrers, dass etwaige Tests schnell vor dem Fahrtantritt bewältigt werden müssen, ist eine kurze Testdauer und eine ausgewogene Schwierigkeit essentiell. Dieses Problem wird im Abschnitt \ref{openChallenges} noch thematisiert.

Parameter \textit{PS02} kann u.a. mit dem vom Albrecht et al. entwickelten System untersucht werden \cite{mobilesmarttracking}. Hier sei es möglich, mithilfe der Kamera eines Smartphones einen Nystagmus des Auges festzustellen. Die Ergebnisse seien sehr vielversprechend. Durch soll ein Augenzittern ist ein verminderndes Sehvermögen des Fahrers durchaus beurteilbar. Anderseits beschreiben Albrecht et al. ebenfalls, dass das aufgezeigte System als Teil einer umfänglicheren Testbatterie, und nicht als einziges Indiz genutzt werden sollte \cite{mobilesmarttracking}. Als weitere Testmaßnahme zum Sehvermögen kann indirekt der Linienverfolgungstest gewählt werden. Eine schlechte Sehleistung könnte bei diesem dazu führen, dass Linienverläufe falsch erkannt werden. Somit könnte man durch ein schlechtes Testergebnis auf eine schlechte Sehleistung folgen. Die Ermittlung des Sehvermögens bleibt jedoch kritisch in diesem Kontext. Zwar gibt es indirekte Methoden zur Bestimmung, jedoch führt selbst die Wiener Testreihe keinen eindeutigen Test zur direkten Messung.

Das räumliche Vorstellungsvermögen (\textit{PS03}) könnte mit dem Adaptiven Dreidimensionalen Würfeltest getestet werden. Laut der Studie von Bennett et al. hat der Test auch eine gute Zuverlässigkeit \cite{cognitivetestsfitnesstodrive}. Jedoch hat der Test eine Länge zwischen 29 und 52 Minuten \footnote{Schuhfried.at. (2018). SCHUHFRIED - A3DW. [online] Verfügbar unter: https://www.schuhfried.at/test/A3DW [Zugriff 1 Feb. 2018]} und wäre in dieser Form nicht einsetzbar. Der Intelligenz-Basis-Funktionen-Test prüft die räumliche Vorstellungskraft ebenfalls indirekt über einzelne Items. Somit kommt es auf die einzelnen Fragen innerhalb dieses Test an, um gezielt den Parameter PS03 zu überprüfen. Jedoch fassen Bennett et al. in ihrer Studie die Aussagekraft eines solchen Intelligenztest als nicht so hoch ein \cite{cognitivetestsfitnesstodrive}. PS03 ist somit mit genannten Test-Methoden eher nicht valide und wieder indirekt zu überprüfen.

\todo[inline]{restliche Parameter}
\todo[inline]{Aufzählen, welche Parameter gut zu messen sind}

\subsection{Kategorie Smart-Tracking Wearables (PW)}