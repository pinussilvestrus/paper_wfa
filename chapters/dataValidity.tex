\section{Analyse der Datenvalidität}
\label{dataValidity}
Nachdem im Abschnitt \ref{parameters} eine Reihe von geeigneten Parametern ermittelt wurde, soll die Qualität der aus den darauf beruhenden Messmöglichkeiten untersucht werden.

\subsection{Kategorie App-Tests Smartphone (PS)}

Die einzelnen Parameter werden im Folgenden nacheinander mithilfe verschiedener Quellen auf Qualität überprüft. Auf die Aufgabenstellung einzelner Tests wird in dieser Arbeit nicht eingegangen. Die SCHUHFRIED GmbH verweist auf den einzelnen Unterseiten der im folgenden genannten Tests, dass sie Rasch-Modell-konform seien. Damit ist gemeint, dass die Testergebnisse ``spezifisch objektive, d.h. item- und personenunabhängige Testresultate" \footnote{Spektrum.de. (2018). Rasch-Modell - Lexikon der Psychologie. [online] Verfügbar unter: http://www.spektrum.de/lexikon/psychologie/rasch-modell/12443 [Zugriff 4 Feb. 2018].} erzielen. Dies sei ein sehr wichtiges Kriterium, um die Validität von derartigen, psychologischen Tests zu bewerten.

Laut der Studie von Kramm ist der 1992 entwickelte Cognitrone-Test (\textit{PS01}) zuverlässig in der Bestimmung der Aufmerksamkeit - und Konzentrationsleistung \cite{studieaufmerksamkeitstests}. Mit einer Dauer von ca. 6-10 Minuten sei er zudem relativ kurz. Die Studie von Neelima schätzt die Vertrauenswürdigkeit der Test-ergebnisse eher durchschnittlich ein \cite{indiaassessment}. Die Schwierigkeit des Tests sei mit zunehmendem Alter jedoch als sehr hoch einzuschätzen. In Hinsicht zur Situation des Fahrers, dass etwaige Tests schnell vor dem Fahrtantritt bewältigt werden müssen, ist eine kurze Testdauer und eine ausgewogene Schwierigkeit essentiell. Dieses Problem wird im Abschnitt \ref{openChallenges} noch thematisiert.

Parameter Sehvermögen (\textit{PS02}) kann u.a. mit dem vom Albrecht et al. entwickelten System untersucht werden \cite{mobilesmarttracking}. Hier sei es möglich, mithilfe der Kamera eines Smartphones einen Nystagmus des Auges festzustellen. Die Ergebnisse seien sehr vielversprechend. Durch die Erkennung eines Augenzitterns ist ein verminderndes Sehvermögen des Fahrers durchaus beurteilbar. Anderseits beschreiben Albrecht et al. ebenfalls, dass das aufgezeigte System als Teil einer umfänglicheren Testbatterie, und nicht als einziges Indiz genutzt werden sollte (Ebd.). Als weitere Testmaßnahme zum Sehvermögen kann indirekt der Linienverfolgungstest gewählt werden. Eine schlechte Sehleistung könnte bei diesem Verfahren dazu führen, dass Linienverläufe falsch erkannt werden. Somit könnte man durch ein schlechtes Testergebnis auf eine schlechte Sehleistung schlussfolgern. Die Ermittlung des Sehvermögens bleibt jedoch kritisch in diesem Kontext. Zwar gibt es indirekte Methoden zur Bestimmung, jedoch führt selbst die Wiener Testreihe keinen eindeutigen Test zur direkten Messung.

Das räumliche Vorstellungsvermögen (\textit{PS03}) könnte mit dem Adaptiven Dreidimensionalen Würfeltest überprüft werden. Laut der Studie von Bennett et al. hat der Test eine hohe Zuverlässigkeit \cite{cognitivetestsfitnesstodrive}. Jedoch hat der Test eine Länge zwischen 29 und 52 Minuten\footnote{Schuhfried.at. (2018). SCHUHFRIED - A3DW. [online] Verfügbar unter: https://www.schuhfried.at/test/A3DW [Zugriff 1 Feb. 2018]} und wäre in dieser Form nicht einsetzbar. Der Intelligenz-Basis-Funktionen-Test prüft die räumliche Vorstellungskraft ebenfalls indirekt über einzelne Items. Somit kommt es auf die einzelnen Fragen innerhalb dieses Tests an, um gezielt den Parameter \textit{PS03} zu überprüfen. Allerdings schätzen Bennett et al. in ihrer Studie die Aussagekraft eines solchen Intelligenztest als nicht so hoch ein \cite{cognitivetestsfitnesstodrive}. \textit{PS03} ist somit mit genannten Test-Methoden eher nicht valide und nur indirekt zu überprüfen.

Das Gedächtnis (\textit{PS04}) kann mithilfe des Visuellen Gedächtnistests überprüft werden. In ähnlichen Ausführungen, wie beispielsweise der Salford Objective Recognition Test, wird ein solcher Test in der Arbeit von Bennett et al. als vertrauenswürdig eingeschätzt \cite{cognitivetestsfitnesstodrive}.

Die Belastbarkeit (\textit{PS05})  kann unter anderem mit dem Wiener Determinationstest überprüft werden. Die Studie von Neelima schätzt die Validtät der Testergebnisse des DT auf gut bis durchschnittlich ein \cite{indiaassessment}. Schuhfried et al. haben eine umfassende Evaluierung dieses Tests vorgenommen und dessen Validität sehr detailliert bestätigt \cite{wiendt}. Der Test habe in seiner Kurform nur eine Dauer von 5 Minuten und keine hohe Komplexität, sei aber trotzdem sehr zuverlässig in der Messung der Belastbarkeit.

Der Parameter der Anpassungsfähigkeit (\textit{PS06}) schließt neben der Anpassungsfähigkeit die verkehrstechnische Entscheidungsfähigkeit mit ein. Für beide Eigenschaften ist der ATAVT entwickelt worden, um Situationen im Straßenverkehr zu präsentieren und die Beobachtungsfähigkeit des Fahrers einzuschätzen. Die Länge des Tests ist auf mindestens 8 Minuten\textsuperscript{\ref{foot:atavt}}  einzuordnen und somit zu lang, um vor Fahrtantritt eingesetzt zu werden. Zudem bewertet Neelima in ihrer Studie die Validität des ATAVT eher durchschnittlich \cite{indiaassessment}.

Der Trail-Making-Test  ist laut der Studie von Bennett et al. eines der international renommiertesten Tests für das Messen der Reaktionsfähigkeit (\textit{PS07}) \cite{cognitivetestsfitnesstodrive}. Ebenfalls haben Baker et al. festgestellt, dass dieser Test sehr häufig in anderen Studien benutzt  und zudem als sehr zuverlässig eingeschätzt wird \cite{reviewofassessmenttests}. Außerdem dauert der Test nur wenige Minuten und wäre somit auf dem Smartphone sehr einfach implementierbar und vor Fahrtantritt schnell ausführbar. Der Wiener Reaktionstest ist eine weitere Möglichkeit, um die Reaktionsfähigkeit zu testen. Er ist mit 5-10 Minuten Testdauer\footnote{Schuhfried.at. (2018). SCHUHFRIED - RT. [online] Verfügbar unter: https://www.schuhfried.at/test/RT [Zugriff 1 Feb. 2018]} relativ kurz. Das einfache, schnelle Drücken von Farbpunkten auf dem Smartphone-Display wäre keine große Implementierungsaufgabe.

Zur Überprüfung der Fähigkeit zur Selbstkontrolle (\textit{PS08}) eines Fahrers werden gängigerweise Fragebögen wie der IVPE-Test (Inventar verkehrsrelevanter Persönlichkeitseigenschaften) genutzt. Das Problem an solchen Fragebögen haben Torner et al. in ihrer Arbeit herausgearbeitet \cite{verfalschbarkeitivpe}. Demnach seien Persönlichkeitstests wie der IVPE-Test leicht fälschbar, da sie vom Fahrer selbst vorgenommen werden und zum Teil unehrliche Antworten gegeben werden. Außerdem sind solche Fragebögen durch ihre zunehmende Dauer schlecht einsetzbar vor dem Fahrtantritt.

Um die Fähigkeit zur persönlichen Organisation (\textit{PS09}) eines Fahrers zu testen, wird oft der psychologische Test 'Tower of London' gewählt. Nach der Studie von Kaller et al. kann das Bewältigen dieses Planungstests auf eine erhöhte Planungsfähigkeit hinweisen \cite{toweroflondon}. Der Test kann jedoch durch seine Schwierigkeit sehr lang dauern.

Mit dem Thema der Erkennung von Alkoholismus (\textit{PS10}) eines Fahrers haben sich gleich mehrere Arbeiten beschäftigt \cite{mobilesmarttracking,sobrietymobiletests}. Dieses Kriterium wurde zudem von gängigen Richtlinien als sehr wichtig eingestuft \cite{begutachtungsrichtlinien,testverfahrenpsychometrischefahreignung,beurteilungskriterien,drivervehiclelicencingagency}. Die von Albrecht et al. aufgezeigte Kamera-App hat gezeigt, dass es sehr gut möglich ist, einen Nystagmus des Auges zu erkennen und darauf auf einen alkoholisierten, und somit nicht fahrtauglichen Zustand des Fahrers zu schließen \cite{mobilesmarttracking}. Des Weiteren gäbe es die Möglichkeit, durch Fragebögen den Umgang eines Fahrers mit Alkohol zu bestimmen, wie zum Beispiel dem Fragebogen zum funktionalen Trinken (FFT). Jedoch haben wie beim Parameter \textit{PS08} solche Fragebögen den Nachteil, als Selbsttest nicht objektiv genug zu sein.

Der letzte Parameter in dieser Kategorie beschäftigt sich mit der Aggression eines Fahrers (\textit{PS11}). Auch hier lassen sich viele Fragebögen einsetzen, wie zum Beispiel der Verkehrsspezifische Itempool (VIP), der nebenbei noch viele weitere verkehrsrelevante Eigenschaften testet und mit ca. 10 Minuten \footnote{Schuhfried.at. (2018). SCHUHFRIED - VIP. [online] Verfügbar unter: https://www.schuhfried.at/test/VIP [Zugriff 3 Feb. 2018]} noch im Rahmen ist. Auch hier stellt das beim Parameter \textit{PS08} und \textit{PS10} genannte Problem der Verfälschbarkeit eine wichtige Rolle.

Zusammenfassend hat die Analyse der ersten Kategorie folgende Parameter inklusive Testverfahren als praxistauglich eingeschätzt:

\begin{flushleft}
	\begin{itemize}
		\setlength\itemsep{2pt}
		\item PS01: Cognitrone (COG) 
		\item PS04: Visueller Gedächtnistest (VISGED) 
		\item PS05: Wiener Determinationstest (DT) 
		\item PS07: Wiener Reaktionstest (RT), Trail-Making-Test (TMT)
		\item PS09: Tower of London (TOL) 
		\item PS10: App-Test nach Albrecht et al. \cite{mobilesmarttracking} 
	\end{itemize}
\end{flushleft}
Mit Abstrichen könnte jedoch auch der Verkehrsspezifische Itempool (VIP) genutzt werden, da dieser viele verschiedene Parameter abdeckt. Bei allen genannten Testverfahren bleibt in einer weiteren praktischen Anwendung zu untersuchen, inwieweit sich diese anwenden lassen.

\subsection{Kategorie Smart-Tracking Wearables (PW)}

Ein großes Problem in der Evaluation der einzelnen Messsensoren an Wearables ist die Größe des Marktes an derartigen Produkten. Viele der im Folgenden genannten Studien beziehen sich auf einzelne Geräte. Im Rahmen dieser Arbeit soll eine allgemeine Möglichkeitsabschätzung getätigt werden. Eine detaillierter Vergleich der einzelnen Geräte ist nicht Inhalt.

Die allgemeine körperliche Fitness eines Fahrers (\textit{PW01}) kann zum Beispiel über die tägliche Aktivität des Fahrers gemessen werden. In nahezu allen Wearables gibt es Schrittzähler, die daraus Aussagen über die körperliche Verfassung der Nutzers generieren können.  El-Amrawy et al. haben kleine Unterschiede in der Genauigkeit von insgesamt 17 getesteten Geräten bestimmt \cite{wearabletracking}. Die Genauigkeiten zur tatsächlichen Schrittzahl seien zwischen 99.1\% (MisFit Shine) und 79.8 \% (Samsung Gear 2) ermittelt worden, was insgesamt auf eine sehr hohe Validität schließen lässt. Case et al. haben in ihrer Studie unter anderem die Genauigkeit des Beschleunigungssensors und Pedometers von Wearables untersucht \cite{studyaccuracysmartphoneapplications}. Mit Ausnahme eines einzigen Gerätes sei die Qualität sehr hoch. Parameter \textit{PW01} lässt sich außerdem durch die Überprüfung der Sauerstoffsättigung im Blut analysieren. Hierbei haben Preejith et al. untersucht, dass sich diese durch einen Fingersensor gut bestimmen lässt \cite{spo2oxygen}. 

Eine weitere wichtige körperliche Eigenschaft eines Fahrers ist die Balance (\textit{PW02}). Diese könnte mittels Gyroskop, Accelerometer und Gleichgewichtssensor in Wearables gemessen werden. Jedoch haben Mancini und Horak in ihrer Arbeit offen gelegt, dass Gleichgewichtsstörungen nur indirekt mithilfe dieser Sensoren zu messen und zusätzliche Berechnungen nötig sind \cite{balancewearables}. Shany et al. haben in ihrer Arbeit zusammengefasst, dass das Potenzial von solchen Sensoren in Wearables durchaus sehr groß ist, um die Balance eines Nutzers zu bestimmen \cite{sensorbasedfalls}. Jedoch geben auch sie an, dass viele Berechnungen nötig sind, um den Parameter einwandfrei bestimmen zu können. Eine weitere sehr interessante Möglichkeit, die Balance und Koordination zu messen, ist der Einsatz von Smartglasses. Nach einer Studie von Salisbury et al. hätte diese Einschätzung eine Genauigkeit von bis zu 95 \% im Vergleich zu gängigen Balancemessungen und sei somit sehr vertrauenswürdig \cite{smartglasses}.

Die stressliche Belastung bzw. Aufregung eines Fahrers vor Fahrtantritt (\textit{PW03}) kann mithilfe eines Puls - oder Herzfrequenzsensors, der bereits in vielen Wearables integriert ist, überprüft werden. Jo et al. haben herausgearbeitet, dass die Qualität verschiedener Fitness-Armbänder im Bereich Herzfrequenzmessung auseinander geht \cite{biofeedbackwearables}. Einige Modelle wiesen eine hohe Validität auf, andere nicht. Ähnliche Ergebnisse erzielte Dooley in ihrer Arbeit, nach der die Validität verschiedener Geräte sehr unterschiedlich ausfällt \cite{selfmonitoringheartrate}. El-Amrawy et al. haben jedoch in ihrer Arbeit herausgefunden, dass die Genauigkeit der Messungen zwischen 99.9\% (Apple Watch) und 92.8\% allgemein sehr hoch sei \cite{wearabletracking}. 

Mithilfe der Schlafüberwachung, die an vielen Fitness-Armbändern implementiert ist, kann indirekt der Parameter \textit{PW04} gemessen werden. Das Systematische Literaturreview von Evenson et al. hat herausgearbeitet, dass die Validität und Vertrauenswürdigkeit dieser Schlafüberwachung bei vielen Geräten sehr hoch ist \cite{reviewconsumerwearables}.

Zusammenfassend hat die Recherche für diese Kategorie ergeben, dass alle Parameter durchaus gute Ergebnisse liefern, um die körperlichen Eigenschaften eines Fahrers vor Fahrtantritt zu prüfen. Der große Vorteil gegenüber der ersten Kategorie ist hier, dass die Daten in Echtzeit genutzt werden können und keine minutenlangen Tests absolviert werden müssen.
