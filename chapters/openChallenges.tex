\section{Diskussion und Offene Fragestellungen}
\label{openChallenges}

Neben den in den vorhergehenden Kapiteln genannten Ergebnissen haben sich eine Reihe von Problemen herauskristallisiert. Aus diesen lassen sich weitere interessante Forschungsfragen in diesem Themenbereich formulieren, welche im Folgenden erläutert werden sollen.

Im Abschnitt \ref{concept} wurde ein Konzept für die automatisierte Evaluation der Fahrtauglichkeit vorgestellt. Problematisch wird die Anwendung dann, wenn man die Anwenderseite betrachtet. Die Aussage, dass ein Fahrer die Tests zu jeder Zeit freiwillig vornehmen wird, ist als sehr kritisch einzustufen. Es bleibt zu betrachten, wie hoch die Bereitschaft auf Seiten der Fahrer ist, dass aufgezeigte System einzusetzen. Abhilfe könnte hier eine Pflichteinrichtung des Systems im Fahrzeug sein. Denkbar wäre, dass das Fahrzeug erst in Betrieb genommen werden kann, wenn das System die Fahrtauglichkeit des Fahrers bestätigt. Dies wäre beispielsweise bei bereits alkoholauffälligen Fahrern denkbar, muss aber weiter untersucht werden. Ein weiterer Punkt ist die Qualität der Daten. Zwar wurden im Abschnitt \ref{dataValidity} die Parameter ausgefiltert, die leicht verfälschbar sind. Dennoch können unter Drucksituationen oder in Notsituationen falsche Testergebnisse in der Parameter-Kategorie App-Tests vorliegen. Weiterhin muss untersucht werden, wie kurz die genannten Tests sein dürfen, um vorm Fahrtantritt benutzt zu werden. Des weiteren kann eine hohe Schwierigkeit der Tests dazu führen, dass die Frustration eines Fahrers sehr hoch wird. 

Weiterhin ist der Datenschutz eine wichtige Fragestellung, die betrachtet werden sollte. Zum einen müssen psychologische Fahrerdaten und Testergebnisse vor Missbrauch geschützt werden \cite{beurteilungskriterienleipzig}. Ebenfalls muss diskutiert werden, wie die körperlichen Daten aus dem Smart Tracking privat bleiben \cite{securityprivacyfitnesstracking}. Das im Abschnitt \ref{concept} dargestellte Konzept muss dementsprechend erweitert werden.

Des weiteren hat die Analyse der Datenvalidität gezeigt, dass die Ergebnisse der Sensoren zur Einschätzung der körperlichen Eigenschaften sehr gut sind. Weiterhin muss aber konzipiert werden, auf welcher Grundlage die Einschätzung getätigt werden müssen. Ein Punktesystem wie in der Anwendung \textit{DriveSafe} ist denkbar \cite{drivesafe}. Hinzu können dann auch Daten aus Smartphone-Sensoren genommen werden. Beispielsweise kann der eingebaute Schrittzähler ebenfalls Aussagen über die körperliche Fitness des Fahrers treffen \cite{validationphysicalactivitytracking,bewegungserkennungsensoren}. In dieser Arbeit wurde die Evaluation nur auf Grundlage der Sensoren in Wearables vorgenommen.

Ebenfalls nicht Teil dieser Arbeit war ein detaillierter Vergleich der einzelnen Geräte. Die im Abschnitt \ref{dataValidity} dargestellten Studien beschäftigten sich teilweise mit einzelnen Modellen, in dieser Arbeit wurden diese Ergebnisse verallgemeinert, um generelle Aussagen über die Datenvalidität der Sensoren zu treffen. Sicherlich muss beachtet werden, welche Wearables in einem etwaigen Prototypen genutzt werden und wie zuverlässig die gewonnenen Daten im Einzelfall sind.
