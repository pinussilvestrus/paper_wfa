\section{Einleitung}
\label{introduction}

\todo[inline]{
	Aufbau Introduction:
	- Was ist das generelle Problem und warum ist es wichtig?
	– Was ist das spezifische Problem und warum ist es interessant? – Was sind Unterschiede zu vorherigen Arbeiten?
	– Welche Motivation gab es für diesen Artikel?
	– Was sind die Ziele und Beiträge und warum sind diese neu?
	– Was sind die Hauptergebnisse?
	– Was ist der generelle Ansatz?
	– Aufbau des Papers}
Das Entstehen von Verkehrsunfällen, die auf fahruntüchtige Verkehrsteilnehmer zurückzuführen sind, ist ein großes Problem für die Sicherheit im Straßenverkehr. Es gibt zahlreiche Richtlinien und psychologische Tests, um die Fahrtauglichkeit eines Fahrers zu überprüfen \cite{drivervehiclelicencingagency,testverfahrenpsychometrischefahreignung,begutachtungsrichtlinien,beurteilungskriterien}\todo{mehrere Zitate in einer Klammer?}. Diese bestehen jedoch aus langwierigen Untersuchungen, so dass eine finale Einschätzung erst nach einer bestimmten Zeit getroffen werden kann.  Eine schnelle Entscheidung vor Fahrtantritt ist damit ohne Weiteres nicht möglich.

Die Integration von Smartphones in Fahrerassistenzsystemen ist ein bedeutsames Gebiet im Bereich Industrie 4.0. Ebenso finden Wearable Devices immer mehr Beachtung im Privatleben einzelner Personen. Wo 2014 noch 9 Millionen Fitnessarmbänder weltweit verkauft wurden, soll die Stückzahl bis 2019 auf 99 Millionen steigen \cite{wearabletracking}. 

Die nachfolgende Arbeit soll sich somit mit der Frage beschäftigen, inwieweit sich die Fahrtauglichkeit eines Fahrers vor dem Fahrtantritt mithilfe von Smartphones und Wearables messen lassen kann.  \todo{Weiter ausführen!}

Im Kapitel \ref{relatedWork} soll es zunächst darum gehen, geltende Richtlinien aufzuführen, die im Zusammenhang mit der Feststellung der Fahreignung stehen. Außerdem wird aufgeführt, welche verwandten Arbeiten im Bereich der Softwareentwicklung bereits veröffentlicht wurden. Danach werden im Kapitel \ref{parameters} geeignete Parameter für das Problem Fahrtauglichkeit sowie mögliche Messmethoden ermittelt. Daraus gehend werden im Abschnitt \ref{dataValidity} jene Messmöglichkeiten auf Qualität geprüft und ein erstes Konzept zur automatischen Erfassung der Parameter im Abschnitt \ref{concept} beschrieben. Anschließend werden die gewonnen Ergebnisse im Kapitel \ref{evaluation} diskutiert und offene Fragestellungen im Kapitel \ref{openChallenges} aufgeführt. Abschließend werden alle Ergebnisse dieser Arbeit im Kapitel \ref{conclusion} zusammengetragen.