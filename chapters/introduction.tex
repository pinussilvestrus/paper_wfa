\section{Einleitung}
\label{introduction}

Das Entstehen von Verkehrsunfällen, die auf fahruntüchtige Verkehrsteilnehmer zurückzuführen sind, ist ein großes Problem für die Sicherheit im Straßenverkehr. Es gibt zahlreiche Richtlinien und psychologische Tests, um die Fahrtauglichkeit eines Fahrers zu überprüfen \cite{drivervehiclelicencingagency,testverfahrenpsychometrischefahreignung,begutachtungsrichtlinien,beurteilungskriterien}. Diese bestehen jedoch aus langwierigen Untersuchungen, so dass eine finale Einschätzung erst nach einer bestimmten Zeit getroffen werden kann.  Eine schnelle Entscheidung vor Fahrtantritt ist damit ohne Weiteres nicht möglich.

Die Integration von Smartphones in Fahrerassistenzsystemen ist ein bedeutsames Gebiet im Bereich Industrie 4.0. Ebenso finden Wearable Devices immer mehr Beachtung im Privatleben des Menschen. Wo 2014 noch 17 Millionen Fitnessarmbänder weltweit verkauft wurden, soll die Stückzahl bis 2019 auf 99 Millionen steigen \cite{wearabletracking}. 

Die nachfolgende Arbeit soll sich somit mit der Frage beschäftigen, inwieweit sich die Fahrtauglichkeit eines Fahrers vor dem Fahrtantritt mithilfe von Smartphones und Wearables messen lassen kann. Damit wären schnellere Entscheidungen möglich, womit Schäden an Fahrzeug und Mensch im Straßenverkehr effizienter vorgebeugt werden können. Die aktuelle Technologie bei mobilen Applikationen, sowie deren eingebaute Sensoren macht eine schnelle Auswertung möglich. So wird in dieser Arbeit ein Konzept dargestellt, in welchem relevante Daten der internen Sensoren von Smartphones und Wearables, sowie Testergebnisse mittels psychologischen App-Tests, gebündelt ausgewertet werden können. 

Im Weiteren ist es auch interessant zu beobachten, wie hoch die Akzeptanz von Fahrtests bei Fahrern ist. Die Studie von Wolbers et al. zeigt auf, dass eine Fahrbeurteilung mittels aktueller Technik positiver eingeschätzt wird, als durch einfache psychologische Tests \cite{interaktivefahrsimulation}.

Im Kapitel \ref{relatedWork} soll es zunächst darum gehen, geltende Richtlinien aufzuführen, die im Zusammenhang mit der Feststellung der Fahreignung stehen. Außerdem wird aufgeführt, welche verwandten Arbeiten im Bereich der Softwareentwicklung bereits veröffentlicht wurden. Danach werden im Kapitel \ref{parameters} geeignete Parameter für das Problem Fahrtauglichkeit sowie mögliche Messmethoden ermittelt. Daraus folgend werden im Abschnitt \ref{dataValidity} jene Messmöglichkeiten auf Qualität geprüft und ein erstes Konzept zur automatischen Erfassung der Parameter im Abschnitt \ref{concept} beschrieben. Anschließend werden die gewonnenen Ergebnisse im Kapitel \ref{openChallenges} diskutiert und daraus resultierende, offene Fragestellungen aufgeführt. Abschließend werden alle Ergebnisse dieser Arbeit im Kapitel \ref{conclusion} zusammengetragen.