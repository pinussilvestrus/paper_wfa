\section{Einleitung}
\label{introduction}
Das Entstehen von Verkehrsunfällen, die auf fahruntüchtige Verkehrsteilnehmer zurückzuführen sind, ist ein großes Problem für die Sicherheit im Straßenverkehr. Es gibt zahlreiche Richtlinien und psychologische Tests, um die Fahrtauglichkeit eines Fahrers zu überprüfen \cite{drivervehiclelicencingagency,testverfahrenpsychometrischefahreignung,begutachtungsrichtlinien,beurteilungskriterien}\todo{mehrere Zitate in einer Klammer?}. Diese bestehen jedoch aus langwierigen Untersuchungen, so dass eine finale Einschätzung erst nach einer bestimmten Zeit getroffen werden kann. \todo{wichtig (vor Fahrtantritt)! noch später darauf eingehen}  Eine schnelle Entscheidung vor Fahrtantritt ist damit ohne Weiteres nicht möglich.

Die Integration von Smartphones in Fahrerassistenzsystemen ist ein bedeutsames Gebiet im Bereich Industrie 4.0. Ebenso finden Wearable Devices immer mehr Beachtung im Privatleben einzelner Personen. Wo 2014 noch 9 Millionen Fitnessarmbänder weltweit verkauft wurden, soll die Stückzahl bis 2019 auf 99 Millionen steigen \cite{wearabletracking}. 

Die nachfolgende Arbeit soll sich somit mit der Frage beschäftigen, inwieweit sich die Fahrtauglichkeit eines Fahrers vor dem Fahrtantritt mithilfe von Smartphones und Wearables messen lassen kann. 
\todo[inline]{Inhalte der Kapitel zusammenfassen}