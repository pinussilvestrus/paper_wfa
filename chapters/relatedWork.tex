\section{Hintergrund und verwandte Arbeiten}
\label{relatedWork}
\todo[inline]{Abschnitt vielleicht doch einfach nur 'Related Works' nennen?}
\todo[inline]{Klare Abgrenzung zu den Related Work}
Dieser Abschnitt beschäftigt sich mit der Aufbereitung artverwandter Arbeiten zum Thema Einschätzung der Fahrtauglichkeit. Zuerst wird ein Überblick über gängige Richtlinien in der Verkehrspsychologie gegeben. Anschließend wird eine kurze Übersicht über bereits entwickelter Software in diesem Bereich aufgeführt. 

\subsection{Richtlinien zur Fahrtauglichkeit} 

Grundlage für den Begriff der Fahrtauglichkeit sind in der Bundesrepublik Deutschland \$ 11 und \$ FeV\footnote{\label{foot:fev}Verordnung über die Zulassung von Personen zum Straßenverkehr (Fahrerlaubnis-Verordnung - FeV)} sowie Anhang III der aktuellen Führerscheinrichtlinie 2006/126/EG in der EU. In beiden Verordnungen werden jeweils körperliche sowie geistliche Eigenschaften des Fahrers zur Einschätzung gefordert. Ein besonderes Maß erhält die Beurteilung der Sehfähigkeit. Nachfolgende genannte Richtlinien haben die beiden Verordnungen als Grundlage.

Die Begutachtungsleitlinien zur Kraftfahreignung der Bundesanstalt für Straßenwesen \cite{begutachtungsrichtlinien} dienen der differenzierten Einschätzung der Fahrtauglichkeit. Sie geben einen Überblick über notwendige körperliche und geistige Fähigkeiten eines Fahrers im Straßenverkehr.
Auf Grundlage dessen beschäftigt sich die Arbeit von Poschadel und Falkenstein \cite{testverfahrenpsychometrischefahreignung} mit einer Reihe von Tests zur psychologischen Fahreinschätzung. Todo..

\subsection{Softwaretechnische Ansätze}