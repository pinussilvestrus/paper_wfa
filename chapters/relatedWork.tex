\section{Hintergrund und Related Works}
\label{relatedWork}
Zuerst wird ein Überblick über gängige Richtlinien in der Verkehrspsychologie gegeben. Anschließend wird eine kurze Übersicht über bereits entwickelter Software, schwerpunktmäßig mobile Apps, in diesem Bereich aufgeführt. 

\subsection{Richtlinien zur Fahrtauglichkeit} 

Grundlage für den Begriff der Fahrtauglichkeit sind in der Bundesrepublik Deutschland \S 11 und \S 12 FeV\footnote{\label{foot:fev}Verordnung über die Zulassung von Personen zum Straßenverkehr (Fahrerlaubnis-Verordnung - FeV)} sowie Anhang III der aktuellen Führerscheinrichtlinie 2006/126/EG in der EU. In beiden Verordnungen werden jeweils körperliche sowie geistliche Eigenschaften des Fahrers zur Einschätzung gefordert. Ein besonderes Maß erhält die Beurteilung der Sehfähigkeit. Nachfolgend genannte Richtlinien haben die beiden Verordnungen als Grundlage.

Die Begutachtungsleitlinien zur Kraftfahreignung der Bundesanstalt für Straßenwesen dienen der differenzierten Einschätzung der Fahrtauglichkeit \cite{begutachtungsrichtlinien}. Sie geben einen Überblick über notwendige körperliche und geistige Fähigkeiten eines Fahrers im Straßenverkehr.
Auf Grundlage dessen beschäftigt sich die Arbeit von Poschadel und Falkenstein mit einer Reihe von Tests zur psychologischen Fahreinschätzung \cite{testverfahrenpsychometrischefahreignung}. Viel wesentlicher ist in dieser Arbeit aber die Evaluation der vorher in den Begutachtungsleitlinien zur Kraftfahreignung aufgeführten "Testgütekriterien" \cite{begutachtungsrichtlinien}. Als weitere Arbeit in diesem Kontext sind die Beurteilungskriterien von Schubert et al. zu nennen \cite{beurteilungskriterien}. Stellvertretend für die Deutsche Gesellschaft für Verkehrspsychologie werden eine Reihe von Kriterien für einen psychologischen Test zur Fahrtauglichkeit genannt. Vermehrt wird  auf das Fahren unter Alkohol - und Drogeneinfluss, sowie auf vergangene Verkehrsauffälligkeiten des zu Testenden bezogen. Allgemeine und persönliche Eigenschaften finden nur einen kleinen Teil in dieser Richtlinie. 

Die Richtlinie der Driver \& Vehicle Licensing Agency ist der Standard der Fahrtauglichskeitseinschätzung im britischen Raum \cite{drivervehiclelicencingagency}. In ihr werden eine Reihe von Anforderungen an einen Fahrer, dazu verschiedene psychische und körperliche Krankheiten sowie deren Einfluss in der Einschätzung der Fahrtauglichkeit aufgeführt.

Die aufgeführten Richtlinien dienen als Ausgangspositionen bei der Ermittlung praxistauglicher Evaluationsparameter für diese Arbeit. Sie eignen sich außerdem als Grundlage für weitere Arbeiten in der Verkehrspsychologie, auf die zum Teil noch in Kapitel \ref{parameters} verwiesen wird.

\subsection{Softwaretechnische Ansätze}

Bereits in mehreren wissenschaftlichen Arbeiten wurde verschiedene Software beschrieben, die sich mit einzelnen Aspekten der Fahrtauglichkeit beschäftigen. Beispielsweise haben Albrecht et al. in ihrer Arbeit eine App entwickelt, die einen Nystagmus\footnote{\label{foot:nystagmus} Nystagmus: unkontrollierbare, rhythmisch verlaufende Bewegungen eines Organs, am häufigsten die des Auges}  der Augen über einen Eye-Tracking-Kameratest ermitteln kann \cite{mobilesmarttracking}. Damit könne man einen starken Alkoholkonsum, sowie die Einnahme von Betäubungsmitteln durch den Fahrer nachweisen.

Ein ähnliche Herangehensweise hatte bereits Khosravianarab in seiner Arbeit zur Erstellung eines mobilen Testsystems zur Überprüfung auf Trunkenheit \cite{sobrietymobiletests}. Smartphones sollen dabei verwendet werden, um polizeilichen Einsatzkräften in einer Kontrolle eine schnelle und genaue Ermittlung von Alkohol - oder Drogenkonsum eines Fahrers zu ermöglichen. Hierbei wird wiederum der Nystagmus der Augen durch etwaige Eye-Tracking-Kameratests über das Smartphone ermittelt.

Nefzger hat in seiner Arbeit zu der Software \textit{Sensor Platform} ein Konzept für eine kostengünstige Plattform zur Bewertung des Fahrverhaltens mithilfe von Fahrerassistenzsystemen, vorrangig mit Sensoren des Smartphones herausgearbeitet \cite{smartphoneresearchplatform}. Die Nutzung von Smartphones anstatt professioneller Fahrerassistenzsysteme wäre demzufolge finanziell attraktiver, die Datenqualität wäre jedoch deutlich schlechter.

Die Analyse des Fahrverhaltens war ebenfalls Schwerpunkt in der Arbeit von Bo, Jian und Li zur Software \textit{Texive} \cite{texive}. Sie beschäftigen sich sehr detailliert mit der Erkennung, ob ein Fahrer während der Fahrt Textnachrichten mit dem Smartphone schreibt. Hierfür werden die Sensoren des Smartphones genutzt, um festzustellen, ob sich ein Fahrer auf dem Weg zur Fahrertür bewegt, er sich bereits im Auto befindet, oder das Auto sich gerade bewegt.

Weiter haben Bergasa et al. in ihrer Entwicklung zur App \textit{DriveSafe} eine Technology aufgeführt, die nur mithilfe der Sensoren von Smartphones das Fahrverhalten eines Fahrers überprüfen können \cite{drivesafe}. Dabei wird ein Punktesystem vorgestellt, womit das Smartphone Fahrablenkungen oder Stress beim Fahrer erkennt.

Einen Weg, psychische Messparameter in Bewertungen mittels aktueller Technologie einfließen zu lassen, zeigen Campos et al. in ihrer Arbeit auf \cite{drivingsimulations}. Sie beschreiben, wie auf Grundlage von psychologischen Tests, Fahrszenarios im Fahrsimulator generiert werden, um die Fahrtauglichkeit des Fahrers in solchen Situationen zu testen. Einen ähnlichen Ansatz verfolgen Hirsekorn und Taylar, in dem sie Virtual Reality Anwendung nutzen, um Fahrer in bestimmten Fahrsituationen einschätzen zu können \cite{vrapplications}.

Der Schwerpunkt vieler verwandter Arbeiten lag in der Messung einzelner Parameter, wie zum Beispiel dem Alkoholismus \cite{mobilesmarttracking,sobrietymobiletests} sowie dem Verhalten des Fahrers während der Fahrt \cite{smartphoneresearchplatform, texive, drivesafe}. Andere Beiträge beschäftigen sich wiederum mit Technologien, die nicht ohne weiteres am Fahrzeug anzubringen sind, wie zum Beispiel die Fahrsimulation \cite{drivingsimulations, interaktivefahrsimulation, vrapplications}. Der Einsatz von bewährten psychologischen Tests fand nur wenig Anwendung. Diese Arbeit soll nun eine Möglichkeit aufzeigen, verschiedene gewählte Parameter \textit{vor dem Fahrtantritt} mit verschiedenen Techniken, Sensoren und appgestützten Fahrertests zu messen und gebündelt auszuwerten.
