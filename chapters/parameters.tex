\section{Feststellung geeigneter Parameter}
\label{parameters}

Diese Arbeit soll sich im großen Maße mit der Ermittlung geeigneter Parameter für die Einschätzung der Fahrtauglichkeit beschäftigen. Dazu werden im Folgenden verschiedene Parameter aus im Abschnitt \ref{relatedWork} genannten Richtlinien genannt und erläutert. Im nächsten Schritt werden dann Möglichkeiten der Messung dieser Parameter aufgeführt. Daraus wird dann eine Auswahl von gut messbaren Parametern getroffen, auf welche sich dann in den folgenden Teilen dieser Arbeit stärker bezogen wird.
\subsection{Ermittlung der Parameter}
\subsection{Möglichkeit der Messungen}
\todo[inline]{Wichtig: Am Ende zum Ergebnis kommen, dass Zweiteilung sinnvoll ist (siehe Zwischenpräsentation)}
\todo[inline]{Hier würde sich die Tabelle der Parameter - Messung aus der Recherche gut eignen, aber aufspalten in zwei Tabellen (Smarttracking und App-Tests)!}